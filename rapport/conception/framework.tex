\section{Framework}

Le Framework est la solution apporté pour permettre aux développeurs de module d'interagir avec notre application. Pour cela le Framework devra proposer différentes solutions d'accès à des méthodes facile d'utilisation et permettant de répondre à l'ensemble de leurs besoins. \\


\subsection{Gestion des fichiers}

Les développeurs ont dans un premier temps besoin d'utiliser des fichiers d'entrées et de sortie. Il leur faut donc un accès à la base de donné d'utilisateur pour avoir accès à leur fichier, et aussi savoir quel utilisateur lance le module. De plus le module à besoin de sauvegarder un fichier de sortie. Donc il faudra fournir différentes méthodes :\\
\begin{itemize}
\item une retournant l'utilisateur connecté sur la page en cours
\item une retournant le chemin un objet fichier sélectionné par un utilisateur dans sa base de donnée
\item une permettant de sauvegarder un fichier sur le serveur et la base de donnée.
\end{itemize}

\subsection{Gestion des taches}

Pour permettre un meilleur fonctionnement de l'application, il serait utile d'utiliser la parallélisation des taches. Pour cela il faut avoir accès à un gestionnaire de taches. Ce gestionnaire de taches doit donc proposer différentes méthodes : \\
\begin{itemize}
\item ajout et lancement d'une tache
\item consultation des taches en cours d'un utilisateur
\item consultation de l'état des taches d'un utilisateur
\item suppression d'une tache
\end{itemize}


\subsection{Vue}

Le framework doit aussi fournir une certaine facilité à inclure les page développées pour les modules. De plus dans un souci d'esthétisme et d'ergonomie les pages des modules doivent s'inclure dans une page type.